% !TeX spellcheck = pl_PL
\documentclass{classrep}
\usepackage[utf8x]{inputenc}
\usepackage[a4paper, left=1.5cm, right=1.5cm, top=1.0cm, bottom=1.5cm, headsep=0.5cm, headheight=13pt]{geometry}
\usepackage{amsmath, amsthm, amssymb, amsfonts}
\usepackage{graphicx}
\usepackage{float}
\usepackage{listings}
\usepackage{caption}
\usepackage{lipsum} % for dummy text only


\lstdefinelanguage{VHDL}{
	morekeywords={
		library,use,all,entity,is,port,in,out,end,architecture,of,if,downto,process,when,then,elsif,else,
		begin,and,LIBRARY,USE,ALL,ENTITY,IS,PORT,IN,OUT,END,ARCHITECTURE,OF,BEGIN,AND
	},
	morecomment=[l]--
}

\usepackage{xcolor}
\colorlet{keyword}{blue!100!black!80}
\colorlet{comment}{green!90!black!90}
\lstdefinestyle{vhdl}{
	language     = VHDL,
	basicstyle   = \ttfamily,
	keywordstyle = \color{keyword}\bfseries,
	 stringstyle=\ttfamily\color{red!50!brown},
	commentstyle = \color{comment}
}
\lstset{language=VHDL,style=vhdl,literate=%
	*{0}{{{\color{red!20!violet}0}}}1
	{1}{{{\color{red!20!violet}1}}}1
	{2}{{{\color{red!20!violet}2}}}1
	{3}{{{\color{red!20!violet}3}}}1
	{4}{{{\color{red!20!violet}4}}}1
	{5}{{{\color{red!20!violet}5}}}1
	{6}{{{\color{red!20!violet}6}}}1
	{7}{{{\color{red!20!violet}7}}}1
	{8}{{{\color{red!20!violet}8}}}1
	{9}{{{\color{red!20!violet}9}}}1}
\lstset{basicstyle=\ttfamily\footnotesize,breaklines=true}
\lstset{framextopmargin=4pt,frame=top,frame=bottom,captionpos=t}
\DeclareCaptionFormat{listing}{\rule{\dimexpr\textwidth\relax}{0.4pt}\par\vskip1pt#1#2#3}
\captionsetup[lstlisting]{format=listing,singlelinecheck=false, margin=0pt, font={sf},labelsep=space,labelfont=bf}


\usepackage{hyperref} % musi być na końcu
\hypersetup{pdfborder={0 0 0 0}}
%%%%%%%%%%%%%
%%%%%%%%%%%%
%%%%%%%%


\studycycle{Elektronika i Telekomunikacja, studia dzienne, mgr II st.}
\coursesemester{I}

\coursename{Programowalne układy cyfrowe}
\courseyear{2014/2015}

\courseteacher{mgr. Tomaszewski Grzegorz}
\coursegroup{poniedziałek, 14:00}

\author{
  \studentinfo{Witold Olechowski}{127517} \and
  \studentinfo{Tomasz Marecik}{127374}
}

\title{Zadanie : Projekt licznika rewersyjnego.}

%\lstinputlisting{block1.vhdl}
%\vskip 2\baselineskip

\begin{document}
\maketitle

\section{Cel ćwiczenia:}
% tu tekst
Używając języka VHDL zrealizować projekt licznika rewersyjnego dla dwóch możliwych wariantów: 

- stan licznika wyświetlany na dwóch wyświetlaczach siedmiu-segmentowych, zliczanie układu w górę
oraz w dół sterowane za pomocą przełącznika suwakowego SW0, przycisk BUTTON1 źródłem
zliczania impulsów

- stan licznika wyświetlany na dwóch wyświetlaczach siedmiu-segmentowych, zliczanie układu w górę
po wciśnięciu przycisku BUTTON1, natomiast zliczanie układu w dół po wciśnięciu przycisku BUTTON0


\section{Pierwszy wariant realizacji zadania}
\label{sec:pierw} % tworzy etykiete do ktorej sie mozna odwolac 
Realizacja pierwszego wariantu polegała na połączeniu dwóch liczników nodulo 10 do zliczania
impulsów jedności i dziesiątek, odpowiednie przypisanie wejść liczników skutkuje zwiększaniem się
liczb na wyświetlaczu po ustawieniu przełącznika suwakowego w pozycji górnej, natomiast
zmniejszaniu się wartości po ustawieniu przełącznika SW0 w pozycji dolnej. Zliczanie impulsów
odbywa się poprzez wciskanie przycisku BUTTON1.

\begin{figure}[H]  % H - obraz dokladnie w tym miejscu, t - u gory etc..
	\centering
	\includegraphics[width=1.0\linewidth]{blok.png}  % zalacza grafike z rozciagnieciem na cala linie  
	\caption{ Schemat blokowy licznika dla pierwszego wariantu. }
	\label{fig:block_bcd_segment}
\end{figure}

Na listingu \ref{lst_mod10} zamieszczona została, implementacja w jezyku VHDL rewersyjnego licznika modulo 10. 

\lstinputlisting[label=lst_mod10,caption=Rewersyjny licznik modulo 10 ,language=VHDL]{mod10.vhd}

Na listingu \ref{lst_bcd} zamieszczona została, implementacja w jezyku VHDL dekodera BCD na kod wskaznika 7-segmentowego. 

\lstinputlisting[label=lst_bcd,caption=Dekoder bcd na 7 segment ,language=VHDL]{decod.vhd}


\subsection{Przebiegi czasowe układu:}
Na rysunkach \ref{fig:symhex0}, \ref{fig:symhex1}, \ref{fig:symhex2}, \ref{fig:symhex3} zostały przedstawione przebiegi czasowe potwierdzające poprawne działanie
licznika rewersyjnego dla pierwszego wariantu. Przycisk BT1 generuje kolejne impulsy zliczające,
natomiast przełącznik suwakowy SW1 steruje kierunkiem zliczanych impulsów (góra, dół).


\begin{figure}[H]
	\centering
	\includegraphics[width=1.0\linewidth]{up_down_1.png}
	\caption{Przebiegi czasowe – zliczanie w dół (SW0=0)}
	\label{fig:symhex0}
\end{figure}



\begin{figure}[H]
	\centering
	\includegraphics[width=1.0\linewidth]{up_down_2.png}
	\caption{Przebiegi czasowe – zliczanie w dół (SW0=0 $\longrightarrow$ SW0=1) zmiana pozycji przełącznika suwakowego SW0 przy skrajnych wartościach	}
	\label{fig:symhex1}
\end{figure}



\begin{figure}[H]

S	\centering
	\includegraphics[width=1.0\linewidth]{up_down_inv_1.png}
	\caption{Przebiegi czasowe – zliczanie w górę (SW0=1) }
	\label{fig:symhex2}
\end{figure}


\begin{figure}[H]
	\centering
	\includegraphics[width=1.0\linewidth]{up_down_inv_2.png}
	\caption{Przebiegi czasowe – zliczanie w górę (SW0=1 $\longrightarrow$ SW0=0) zmiana pozycji przełącznika suwakowego SW0 przy skrajnych wartościach }
	\label{fig:symhex3}
\end{figure}

\section{ Drugi wariant realizacji projektu licznika}

Realizacja drugiego wariantu polegała na połączeniu dwóch liczników modulo 10 do zliczania
impulsów jedności i dziesiątek, odpowiednie przypisanie wejść liczników skutkuje odpowiednio
zwiększaniem się liczb na wyświetlaczu po wciśnięciu przycisku BUTTON1, natomiast zmniejszaniu się
wartości po wciśnięciu przycisku BUTTON0.

Do poprawnego działania drugiego wariantu konieczna była modyfikacja schematu blokowego co
widzimy na Rysunku \ref{fig:block1}. Właściwe działanie bloku mod10\_1 (lst. \ref{lst_mod10_1}) wynika z odpowiednich modyfikacji
kodu w języku VHDL.\\

\begin{figure}[H]  % H - obraz dokladnie w tym miejscu, t - u gory etc..
	\centering
	\includegraphics[width=1.0\linewidth]{blok1.png}  % zalacza grafike z rozciagnieciem na cala linie  
	\caption{ Schemat blokowy licznika dla drugiego wariantu. }
	\label{fig:block1}
\end{figure}

%Licznik zostal niewiele zmodyfikowany w stosunku do pierwszego wariantu (list. \ref{lst_mod10}), oraz w wiekszości znajduje się tam komentarz do kodu. 

\lstinputlisting[label=lst_mod10_1,caption=Rewersyjny licznik modulo 10 ,language=VHDL]{mod10_1.vhd}
Przebiegi czasowe dla drugiego wariantu obrazuje Rysunek \ref{fig:symhex4} , naciskanie przycisków BT0 oraz BT1
odpowiada zwiększaniu lub zmniejszaniu się wartości wyświetlanych na wyświetlaczu.
Przebiegi potwierdzają poprawne działanie licznika dla drugiego wariantu realizowanego projektu.

\begin{figure}[H]
	\centering
	\includegraphics[width=1.0\linewidth]{up_down_1_2.png}
	\caption{Przebiegi czasowe dla drugiego wariantu – ciąg impulsów BT0 oraz BT1 i widoczne
	wartości na wyświetlaczach}
	\label{fig:symhex4}
\end{figure}

\section{Wnioski}

Celem wykonywanego dwiczenia było zapoznanie się z zasadą działania licznika, oraz jego Celem wykonywanego dwiczenia było zapoznanie się z zasadą działania licznika, oraz jego budową
koniecznymi do realizacji dwóch wariantów licznika rewersyjnego w oprogramowaniu Quartus II. Przy
wykorzystaniu języka VHDL utworzono nowy bloczek w postaci licznika modulo 10, odpowiednie
połączenie dwóch takich elementów współpracującymi z dekoderami z poprzedniego projektu w
efekcie umożliwiało zaobserwowanie poprawnego działania licznika rewersyjnego. 

Odpowiednio zrealizowany schemat blokowy wraz z funkcjonalnym kodem VHDL umożliwia poprawne działanie
pierwszego wariantu. Utworzony projekt poddawaliśmy testom na poprawność działania,
przeprowadzone symulacje są potwierdzeniem założeń projektowych. Drugi wariant projektu jest
odpowiednią modyfikacją schematu blokowego pierwszej opcji, konieczne było odpowiednie
zaimplementowanie przycisków BT0 oraz BT1 do generowania impulsów zliczających w górę o raz w
dół. Kilka zmian w listingu bloczku modulo 10 zapewniło finalny efekt w postaci poprawnie
działającego układu zliczającego licznika. 

Podobnie jak w pierwszym przypadku po zaprogramowaniu
układu poddawaliśmy go testom na poprawność. Jednym z pojawiających się problemów przy
realizacji projektu było nie poprawne zliczanie impulsów w przypadku zmiany kierunku zliczania dla
skrajnych wartości na wyświetlaczu (np. dla wartości 9, 19, 10, 20 itp.). Pojawiający problem u dało
się jednak wyeliminowywać poprzez odpowiednią modyfikację kodu VHDL bloczku mod\_10.




\begin{thebibliography}{0}
  \bibitem{l2short} John Wiley and Sons Publishers.
    \textsl{Digital Design,} University of California, Riverside, 2007
\end{thebibliography}
\end{document}
